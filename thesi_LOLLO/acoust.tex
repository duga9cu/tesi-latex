\chapter{Elementi di acustica}
\label{sec:acoust}

\section{Introduzione}
	\label{sec:intracustic}
	Nel presente capitolo verranno fornite le nozioni di base riguardanti i principi fisici fondamentali della fisica del suono.\\
	Partendo dalle grandezze fondamentali che caratterizzano l'analisi del fenomeno sonoro, si parler� della descrizione delle diverse onde sonore 
	per arrivare a una descrizione di campo acustico di cui si cerca di fornire una mappa visiva.
	
	
\section{Grandezze fondamentali}
\label{sec:sound}
	
	[...]\\
	
	\subsection{Frequenza e Periodo}
		\label{sec:freq}
		
		
		\begin{equation}
		  \label{eq:freq}
		  T=\frac{1}{f}
		\end{equation}
		
		[...]\\
	
	\subsection{Velocit� di propagazione e lunghezza d'onda}
			[...]\\
			
	\subsection{La scala dei Decibell}
		\label{sec:dB}
		[...]\\
		

\section{Onde acustiche}
	\label{sec:soundwaves}
	I fenomeni acustici consistono essenzialmente di una perturbazione di pressione che si propaga in un mezzo in equilibrio. Ci� che caratterizza il fenomeno 
	� l'entit� di questa perturbazione rispetto a un valore di equilibrio preso come riferimento. Nel caso applicativo pi� frequente, la propagazione nell'aria, 
	si prende come riferimento la pressione atmosferica.\\
	Essendo $P_0$ la pressione di riferimento del mezzo, la \emph{pressione acustica istantanea} viene definita come segue:
	
	\begin{equation}
	\label{ eq:pressure}
	p(t)=p'(t)-P_0
	\end{equation}
	
	dove $p'(t)$ � il valore di pressione atmosferica nell'istante $t$ in un punto dato in cui si vuole misurare la pressione acustica.\\
	
	\subsection{livello di pressione}
	
		Il valore di pressione acustica varia in un campo molto esteso, per questo motivo si usa riferirsi piuttosto al \emph{livello di pressione acustica $L$}
		(o SPL: Sound Pressure Level) definito:
		\begin{equation}
		\label{ eq:SPL}
		L=20*\log\frac{p}{p_0}(dB)
		\end{equation}
		
		dove $p_0$ sia il valore di pressione di riferimento per l'atmosfera, scelto dal sistema internazionale uguale a 20  $\mu Pa$ che corrisponde
		alla soglia di udibilit� a $1000 Hz$ \footnote{	Per una spiegazione pi� dettagliata sui livelli sonori si veda l'appendice ~\ref{sec:levels} }
.\\

	\subsection{Livello equivalente}
	In applicazioni reali, siamo in presenza di sorgenti con un livello sonoro non costante nel tempo di cui occorre valutare la \emph{rumorosit�}. Descrivendo il fenomeno sonoro con la funzione matematica che ne regola l�andamento del livello (per esempio di pressione), otteniamo una valutazione del livello sonoro in un dato istante ma 
questo non ci fornisce un�informazione sulla rumorosit� globale. Se ad esempio avessimo una sorgente che si 
accende ad intermittenza, conoscere esattamente l�andamento del tempo non aiuterebbe nel valutare il 
livello sonoro che la sorgente produce in un determinato tempo. Si definisce quindi un \emph{livello equivalente} che 
si calcola come: 
\begin{equation}
\label{ }
L_{EQ} = 10 \log(\frac{1}{T}\int^T_0 \frac{p^2(t)}{p_0^2}dt)
\end{equation}

		[...]\\
		



\section{Sistema uditivo umano}
	\label{sec:ears}
	
	%%%%%%%%%%%%%%%%%%%%%%%%%%%%%%%% FIGURA ORECCHIO %%%%%%%%%%%%%%%%%%%%%%%%%%%%
\begin{figure}
  \centering
  \includegraphics[width = 8cm]{img/orecchio.jpg}
  \caption{Apparato acustico umano.}
  \label{fig:ear}
\end{figure}
	
	l'apparato uditivo umano, come si evince dalla Figura ~\ref{fig:ear}, � molto complesso e composto da moltissimi elementi dalle svariate funzioni, ognuno dei quali influisce sulla percezione acustica in maniera rilevante e addirittura alcuni degli effetti legati alla sensazione uditiva, che vengono chiamati \emph{psicoaustici}, riguardano la sola interpretazione, da parte del cervello, dei segnali elettrochimici provenienti dall'apparato uditivo.\\
	
	In questa sede ci interessano solamente alcuni effetti di non linearit� dell'apparato uditivo, le quali comportano conseguenze fondamentali nelle modalit� di analisi di un qualsiasi fenomeno sonoro, quali la descrizione mediante suddivisione in \emph{bande frequenziali} e l'introduzione dei \emph{filtri di ponderazione}, che verranno descritti successivamente.\\
	
	
			%%%%%%%%%%%%%%%%%%%%%%%%%% FIGURA COCLEA %%%%%%%%%%%%%%%%%%%%%%%%%%%%
\begin{figure}
  \centering
  \includegraphics[width = 8cm]{img/coclea.jpg}
  \caption{Risposta non lineare della coclea}
  \label{fig:coclea}
\end{figure}
	
		
		
		\subsection{Effetti di non linearit� dell'orecchio umano}
		\label{sec:nonlinearear}
	Uno degli organi sensoriali principali dell'apparato acustico � la \emph{coclea} rappresentata in Figura ~\ref{fig:coclea}; � lei la responsabile di gran parte degli effetti non lineari di cui discuteremo in seguito.\\
	Sezionando la coclea, si trova una sorta di doppia lamina la quale � caratterizzata da una diversa sensiblit� lungo la sua estensione, a seconda delle frequenze di eccitazione del segnale acustico, alla maniera, ad esempio, di una corda o di una frusta. Si osservi nel grafico di  figura ~\ref{fig:coclea} come le basse frequenze interessino la parte terminale mentre le alte frequenze la parte iniziale. Si evince facilmente inoltre che due rumori con bande sovrapposte (in tutto o in parte) si mascherino in modo tale che il segnale di maggiore intensit� copra il segnale pi� debole, a meno che quest'ultimo non abbia una larghezza di banda sufficientemente larga.
		
	%%%%%%%%%%%%%%%%%%%%%%%%%% FIGURA curve isofoniche %%%%%%%%%%%%%%%%%%%%%%%%%%%%
\begin{figure}
  \centering
  \includegraphics[width = 8cm]{img/isophon.jpg}
  \caption{Grafico delle curve isofoniche}
  \label{fig:isophon}
\end{figure}
	
	Per il sopraccitato ed altri motivi che non approfondiremo in questa sede, il sistema uditivo umano presenta una sensibilit� meno accentuata alle frequenze molto basse (poche decine di $Hz$) ed a quelle elevate (oltre i $15kHz$). Inoltre per procurare la stessa sensazione sonora ($phon$) occorrono, a frequenze diverse, livelli di pressioni sonore diverse, allo stesso modo suoni di stessa intensit� ma frequenza diversa vengono percepiti dall�orecchio in modo diverso.\\
	Questi effetti sono riassunti nel grafico in Figura~\ref{fig:isophon}.
	


	
	[...]\\
	
	\subsection{Bande frequenziali }
		\label{sec:band}
		[...]\\
		
	\subsection{Filtri di ponderazione}
		\label{sec:ponderaz}

		%%%%%%%%%%%%%%%%%%%%%%%%%% FIGURA curve isofoniche %%%%%%%%%%%%%%%%%%%%%%%%%%%
\begin{figure}
  \centering
  \includegraphics[width = 8cm]{img/ponderaz.jpg}
  \caption{Grafico delle curve di ponderazione}
  \label{fig:ponderaz}
\end{figure}
	
		come descritto nel paragrafo~\ref{sec:nonlinearear}, la sensibilit� dell�orecchio varia al variare della frequenza. Per tale motivo il livello di pressione $SPL$ che misuriamo in realt� non corrisponde a una reale sensazione acustica, cio� variazioni del valore di SPL non necessariamente corrispondono a uguali variazioni nella percezione acustica (variazioni di \emph{volume}). Per rendere pi� aderente alla sensazione umana e quindi rendere pi� intuitiva la misura di un fenomeno sonoro, occorre utilizzare dei  filtri di \emph{pesatura} o \emph{ponderazione}. Quelli attualmente utilizzati sono rappresentati in Figura~\ref{fig:ponderaz}.
		Analizzando il grafico notiamo le curve pi� importanti:
		\begin{description}
  \item[ filtro di ponderazione �A�]: il pi� comunemente impiegato e il cui andamento si conforma alla risposta dell�orecchio umano a livelli medio-bassi. Il livello misurato con la ponderazione del filtro A viene chiamato dB(A).
  \item[ filtro di ponderazione �C�]: impiegato per rumori molto forti o esplosioni misurate quindi in dB(C).
\end{description} 
		
