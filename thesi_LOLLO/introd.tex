\chapter{Premessa}
\label{sec:introd}

	Il presente progetto di tesi si pone l'obiettivo mappare il campo acustico dinamico di un ambiente (interno o esterno) tramite una sonda, nella fattispecie un \emph{array microfonico}. Il risultato richiesto quindi � quello di ottenere un videoclip composto dalla sovrapposizione di:

\begin{description}
  \item[un videoclip di sfondo] ottenuto da una particolare videocamera, che rappresenti l'ambiente circostante e contenente quindi un informazione visiva. 
  \item[la mappa acustica dinamica] composta di bande colorate rappresentanti i livelli sonori istantanei nel campo acustico.
  \item[eventuali \emph{metadata}] quali le posizioni delle singole capsule microfoniche dell'array, o i valori dei livelli sonori in determinate posizioni di interesse.
\end{description}  

	Si tratta di un \emph{plug-in} scritto per l'ambiente di editing audio \emph{Audacity}\textregistered \footnote{ http://audacity.sourceforge.net/?lang=it}, celebre software \emph{open-source} molto versatile, dalle possibilit� molto ampie per permettere gli utilizzi anche meno convenzionali.
	Il software sviluppato in questa tesi, dialogando con l'\emph{host} \emph{Audacity}\textregistered, acquisisce i dati in ingresso e genera da essi una \emph{mappa acustica}, composta da bande di colore che corrispondono ai diversi livelli sonori attorno all'array sonda, posto nell'ambiente di misura. \\
	Questo tipo di \emph{tool} pu� avere svariate applicazioni tecnologiche in quanto � in grado di rendere visibile i campi sonori i quali contengono invece una informazione di tipo uditivo. 
	La potenza di questa operazione sinestetica di traduzione di un'informazione visiva in una uditiva, risiede principalmente nella maggior apprezzabilit� delle grandi qualit� di definizione spaziale degli array microfonici. \\
	Dal punto di vista applicativo, il sistema sviluppato pu� essere di estrema utilit� nella precisa individuazione di sorgenti sonore, nonch� nella misura delle loro emissioni. 
	Basti pensare per esempio ad ambienti di tipo industriale, nei quali spesso il campo acustico � complesso e generato da molteplici e varie sorgenti che spesso risultano difficili da individuare con precisione. 
	Un altro esempio calzante riguarda gli ambienti molto ristretti come gli abitacoli di automobili o aereoplani il cui confort � una specifica di progetto che attualmente ricopre un notevole interesse.

	In particolare, la principale espansione da me svolta durante questa tesi, � stata quella di rendere il software in grado di generare una mappa \emph{dinamica} riflettendo i cambiamenti del campo acustico attorno alla sonda istante per istante e sovrapponendo questo risultato a un video di sfondo, posto in trasparenza e acquisito mediante telecamere installate appositamente sulle sonde utilizzate e raffigurante l'ambiente stesso di misura.\\

Di seguito, in sintesi, il contenuto:

\begin{description}
  \item[Capitolo~\ref{sec:acoust}] richiami teorici sulla acustica di base e riguardo alcune proprieta fisiche del suono, 
  		ausilio fondamentale a tutta la trattazione successiva.
  \item[Capitolo~\ref{sec:array}] introduzione agli array microfonici come tecnica di misura dei campi acustici 
  		e delle loro propriet� spaziali di direttivit� e direzionalit�.
  \item[Capitolo~\ref{sec:virtualmikes}] introduzione al concetto di microfono virtuale e relativa sintesi, 
  		ovvero possibilit� offerte da questo artificio, nella fattispecie riguardo alla descrizione spaziale di un campo acustico.
 \item[Capitolo~\ref{sec:audacity}] descrizione dell'ambiente software definito dal programma \emph{host} scelto (\emph{Audacity\textregistered}) 
  		e delle librerie esterne utilizzate.  
 \item[Capitolo~\ref{sec:plugindesc} ] breve manuale d'uso per un utilizzatore del \emph{plug-in}. 
 		Possibili analisi e parametri significativi.
 \item[Capitolo~\ref{sec:internalfunct}] dettagli sul funzionamento interno, algoritmi utilizzati e flusso di lavoro.
\end{description}
                     
		    


