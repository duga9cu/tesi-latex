\chapter{Conclusioni}
\label{sec:endings}
	In questo progetto di tesi si � lavorato alla programmazione di un software che fosse in grado di elaborare registrazioni multicanale prefiltrate, provenienti da un array microfonico generico, congiuntamente a una registrazione video dello stesso evento, in modo da generare in output una mappa sonora dinamica che fornisse una indicazione qualitativa dell'evoluzione del campo acustico.
	Come scelta di progetto � stata mantenuta la linea di implementare il software in forma di plug-in \audacity , piattaforma che permette ai suoi \emph{plug-in} o \emph{moduli} di accedere a moltissime propriet� dell'host e dati del workspace, a differenza degli altri standard diffusi di plug-in come per esempio i \emph{VST} e gli \emph{Audiounits}.

	Il plug-in progettato ai fini di questo progetto � funzionante ed altamente interattivo, permettendo diverse tipologie di realizzazione della mappa grazie alla possibilit� di configurare numerose opzioni, quali lo stile della scala di colore, la percentuale di trasparenza della mappa sovrapposta al video di background, l'unit� di misura dei livelli mostrati, i valori di fondo scala, la lunghezza del frame video, la percentuale di overlap tra i singoli frame etc. 
	
	Dal confronto degli output del modulo in oggetto con altri, calcolati dallo script Matlab\textsuperscript{\texttrademark} \; descritto nel documento \cite{spheric-soundfield}, si � inoltre riscontrata una certa coerenza tra i risultati ottenuti; ben sapendo che non � ancora stato inventato un metodo realmente applicabile per tarare una misura effettuata con un array microfonico, si conclude che i risultati ottenuti con l�uso del \emph{plug-in} siano corretti. 

	In futuro, se si riuscisse a mettere a punto questo metodo di taratura per array microfonici, sarebbe possibile utilizzare il plugin in modo molto efficace in campo rumoristico per effettuare misure di livelli sonori.