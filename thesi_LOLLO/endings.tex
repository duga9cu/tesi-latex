\chapter{Conclusioni}
\label{sec:endings}
	Il plug-in progettato in questa sede � funzionante ed altamente interattivo, con svariate tipologie di realizzazione della mappa grazie alla possibilit� di configurare lo stile della scala di colore, la percentuale di trasparenza della mappa sovrapposta al video di background, l'unit� di misura dei livelli mostrati, i valori di fondo scala, la lunghezza del frame video, la percentuale di overlap tra i singoli frame etc.). 
	Dal confronto degli output del modulo in oggetto con altre calcolate dallo script Matlab\textregistered \; descritto nel documento \cite{spheric-soundfield} si � inoltre riscontrata una certa coerenza tra i risultati ottenuti; ben sapendo che non � ancora stato scoperto un metodo realmente applicabile per tarare una misura effettuata con un array microfonico, si desume che i risultati ottenuti con l�uso del \emph{plug-in} siano corretti. 
%Il modulo � stato compilato correttamente per tre piattaforme \emph{Windows}, \emph{Mac OS X} e \emph{Linux} , come da obiettivo della tesi.


figura delle tre piattaforme
%%%%%%%%%%%%%%%%%%%%%%%%%%%%%%%% FIGURA tre PIATTAFORME %%%%%%%%%%%%%%%%%%%%%%%%%%%%
\begin{figure}
  \centering
%  \includegraphics[width =12cm]{img/3win.pdf}
  \caption{Graphic User Interface sotto le tre diverse piattaforme principali: \emph{Windows}, \emph{Mac OS X} e \emph{Linux}.}
  \label{fig:3win}
\end{figure}

 [...]\\
