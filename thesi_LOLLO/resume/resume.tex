\documentclass[10pt,a4paper]{article}

\usepackage{graphicx}
\usepackage{amsmath}
\usepackage[italian]{babel}
\usepackage{color}
\usepackage{colortbl}


%MARGIN SETTINGS
%
\addtolength{\topmargin}{-2.5cm}
\addtolength{\textheight}{4.5cm}
\addtolength{\textwidth}{2cm}
\addtolength{\hoffset}{-1cm}
%
%PDFLatex SETTING
\pdfpagewidth=\paperwidth
\pdfpageheight=\paperheight
%
\title{\textsc{Sintesi di risposte all'impulso multicanale dai risultati di programmi di simulazione acustica}}
\author{}
\date{}


\begin{document}
%
\pagestyle{empty}
%
%TITLE
\begin{center}
  \LARGE UNIVERSIT\`A DEGLI STUDI DI PARMA\\
\vspace{0.2cm}
\Large FACOLT\`A DI INGEGNERIA\\
\vspace{0.3cm}
\normalsize CORSO DI LAUREA SPECIALISTICA\\ IN INGEGNERIA ELETTRONICA\\
\vspace{0.3cm}
\normalsize ANNO ACCADEMICO 2005/2006\\
\vspace{0.3cm}
\rule{4cm}{.1pt}\\
\vspace{0.3cm}
\Large \textsc{Sintesi di risposte all'impulso multicanale\\ dai risultati di programmi di simulazione acustica}
\end{center}
%
%\vspace{0.3cm}
\begin{center}
\begin{tabular}{ll}
\parbox[c]{10cm}{\hspace*{0.8cm}\emph{Relatore :}} & \\
\parbox[c]{10cm}{\hspace*{0.8cm}Chiar.mo Prof. Angelo Farina} & \\
\parbox[c]{10cm}{} & \\
\parbox[c]{10cm}{\hspace*{0.8cm}\emph{Correlatori :}}    & \\
\parbox[c]{10cm}{\hspace*{0.8cm}Dott. Ing. Andrea Capra} & \parbox[c]{10cm}{\emph{Tesi di laurea di:}}\\
\parbox[c]{10cm}{\hspace*{0.8cm}Dott. Paolo Martignon}   & \parbox[c]{10cm}{Simone Campanini}\\
\end{tabular}
%\vspace{0.4cm}
%\rule{4cm}{.1pt}\\
\end{center}
\vspace{0.8cm}

Questo lavoro ha il suo punto di partenza nella richiesta, da parte del Comune di Parma, di un progetto 
di correzione acustica per la chiesa di San Vitale; successivamente si \`e sviluppato in altre due
direzioni: l'analisi acustica di un abitacolo d'automobile e la scrittura di un programma per la
sintesi delle risposte all'impulso. Quindi all'\emph{analisi}, si \`e affiancata l'\emph{elaborazione}
dei dati provenienti dal medesimo software di simulazione del campo sonoro, \emph{Ramsete}, efficace
strumento di progettazione acustica.\\ 
Il progetto \emph{San Vitale} \`e motivato dall'interesse del committente di poter impiegare l'edificio
per concerti di musica antica, caratterizzata da insiemi vocali e strumentali di ridotte dimensioni; si
trattava dunque di rilevare i parametri acustici del locale e di progettare un allestimento tale da 
portare questi ultimi a valori adeguati al genere di eventi che si intendono rappresentare: vista la cubatura
dell'ambiente, si sono ritenuti obiettivi ragionevoli un tempo di riverbero $T_{20}$ non superiore
a 2.2 secondi ed un indice di chiarezza $C_{80}$ tra 0 e 4 dB. In entrambi i casi si tratta di valori medi.\\
Le misure dello stato di fatto hanno mostrato valori di $T_{20}$ abbastanza elevati, nell'ordine, in media, 
di 4 secondi, dunque,
con l'aiuto del software \emph{Ramsete}, si sono studiati una serie di interventi volti a migliorare la
diffusione di onde dirette e riflessioni precoci e ad abbassare il tempo di riverbero. Struttura essenziale
in tutti gli stati di progetto elaborati \`e la camera d'orchestra, alle quali sono state sinergicamente affiancate 
controsoffittature e pannellature laterali, in tutti i casi rimovibili, a tutela dell'ambiente circostante,
il cui utilizzo a fini concertistici non \`e esclusivo. La soluzione che ha prodotto il comportamento 
acustico voluto, per\`o, \`e quella risultata esteticamente pi\`u invasiva, motivo per cui il committente 
ha preferito a questa un allestimento dall'aspetto pi\`u sobrio, costituito dalla sola camera d'orchestra
(visibile in figura~\ref{fig:churchmodel}). 
Per contenere comunque il tempo di riverbero si \`e pensato di collocare dei tendaggi pesanti sulle cappelle 
laterali: cos\`i facendo, le simulazioni hanno restituito un valore medio di $T_{20}$ pari a 2.5 secondi. La
distribuzione dell'indice di chiarezza nell'uditorio ed all'interno della camera \`e risultata adeguata.\\
%
\begin{figure}[htbp]
  \centering
  \includegraphics[width=6cm]{../img/drawings/sv-sdf-wireframe.pdf}
%  \includegraphics[width=6cm]{../img/drawings/sv-final-wireframe-cutted.pdf}
  \includegraphics[width=6cm]{../img/drawings/sv-final-wireframe-cutted-detail.pdf}
  \caption{\emph{Chiesa di San Vitale}: modello tridimensionale e particolare della camera d'orchestra.}
  \label{fig:churchmodel}
\end{figure}
%
Le simulazioni acustiche di un abitacolo appartengono, in realt\`a, ad un progetto pi\`u ampio, il cui scopo
era di valutare una tecnica ibrida di simulazione acustica: le tecniche di \emph{ray} o \emph{beam tracing},
infatti, non possono dare risultati attendibili quando le dimensioni dell'ambiente sono confrontabili con
quelle delle lunghezze d'onda allo studio, da cui l'idea di simulare il comportamento nella banda da 
20 a 500 Hz mediante soluzione
col metodo degli elementi finiti delle equazioni d'onda e la parte restante dello spettro udibile con
\emph{Ramsete}. Nel presente lavoro ci si \`e occupati solo di quest'ultima elaborazione. Cio che si \`e
ottenuto \`e stata una serie di risposte all'impulso sintetiche (una per ogni coppia ricevitore/sorgente), 
prodotte sommando i risultati a spettro parziale delle due tecniche. Nonostante la grande semplificazione
operata nell'assegnamento dei materiali al modello e nonostante la direttivit\`a delle sorgenti si ritenga
largamente approssimata per via dell'indisponibilit\`a dei \emph{baloon} caratteristici degli altoparlanti 
dell'abitacolo, lo spettro delle risposta all'impulso sintetizzata ha mostrato un ottimo accordo con
quelle misurate, adottate come riferimento.\\
Durante lo svolgimento dei progetti di cui sopra, ed in particolare del progetto \emph{San Vitale}, al
termine di ogni sessione di simulazioni sono stati compiuti sia un raffronto dei descrittori acustici
oggettivi con gli obiettivi, sia una ascolto di musica anecoica auralizzata con la risposta all'impulso
del sistema simulato: lo strumento impiegato inizialmente allo scopo \`e stato \emph{AudioConverter},
incluso nel pacchetto \emph{Ramsete}. Pur essendo tale software ampiamente collaudato ed attendibile nel
calcolo dei parametri acustici, ci si \`e immediatamente resi conto dello scarso realismo all'ascolto delle
risposte all'impulso da esso generate, imputabile al particolare algoritmo di sintesi cui fa ricorso;
tale algoritmo, inoltre, provoca la comparsa di innaturali frastagliamenti nei transitori all'esecuzione di 
\emph{Acoustic Quality Test}. Si \`e allora deciso di scrivere un nuovo programma avente come principale
scopo la generazione di risposte all'impulso udibili a partire dai dati prodotti dal simulatore
prive della percezione di innaturalezza cui si faceva riferimento poc'anzi.
%Il progetto di un software di sintesi delle risposte all'impulso a partire dall'equivalente energetico
%calcolato da un programma di previsione acustica come \emph{Ramsete} non \`e un applicazione nuova: in
%particolare il pacchetto \emph{Ramsete} gi\`a da lungo tempo \`e dotato del programma \emph{AudioConverter}
%che svolge esattamente quest'operazione. Pur essendo largamente collaudato ed attendibile nel calcolo dei
%parametri acustici, \emph{AudioConverter} ha almeno un paio di difetti intrinseci all'algoritmo di
%sintesi: lo scarso realismo all'ascolto delle risposte all'impulso prodotte e la comparsa di innaturali
%frastagliamenti nei transitori all'esecuzione di \emph{Acoustic Quality Test} con queste ultime. Si \`e,
%in questa sede, cercato di scrivere un programma, denominato \emph{getIR}, tale da sintetizzare risposte
%all'impulso senza i problemi di cui sopra. 
In estrema sintesi, produrre una risposta all'impulso udibile\footnote{Un 
file \textsf{wav}, sostanzialmente.} a partire dalla matrice di dieci risposte
all'impulso \emph{energetiche}\footnote{Dette \emph{ecogrammi}, uno per ogni banda
d'ottava.} generata da \emph{Ramsete} significa: ridurne il tempo di campionamento
poich\'e \emph{Ramsete} opera, di norma con tempi dell'ordine del millisecondo,
introdurre informazioni di fase altrimenti assenti e compiere un'ulteriore interpolazione in frequenza.
Tutti questi problemi vengono risolti in \emph{AudioConverter} col metodo dell'interpolazione tramite
rumore bianco, infatti il \emph{vuoto} tra un campione di ogni ecogramma ed il successivo che compare 
abbassando il tempo di campionamento viene colmato con
uno spezzone (\emph{burst}) di rumore modulato in ampiezza con la radice quadrata del campione \emph{energetico}
corrispondente; 
il tutto viene infine sommato ed opportunamente corretto, ricavando
cos\`i la risposta all'impulso cercata. Le forti discontinuit\`a presenti nel rumore, per\`o, sono la
causa dell'artificiosit\`a all'ascolto, si \`e quindi pensato di sostituire tale rumore con \emph{burst}
costruiti dalla somma di 12 segnali sinusoidali ognuno con fase casuale le cui lunghezza e frequenza sono funzione 
della banda rappresentata. 
L'intero algoritmo \`e stato successivamente esteso in modo da poter generare anche risposte all'impulso
caratterizzate spazialmente, secondo le tecniche binaurale (2 canali) ed \emph{Ambisonic} (4 canali,
standard \emph{B-Format}); ci\`o \`e stato reso possibile dalle informazioni geometriche sulle prime
riflessioni che, su richiesta dell'utente, \emph{Ramsete} consente di salvare. Per realizzare i moduli
multicanale, infatti, si \`e intervenuti solo nell'algoritmo di generazione della \emph{testa} della
risposta all'impulso, nella quale ogni riflessione \`e modellata o con una $\delta$ \emph{di Dirac} ideale, o
con un impulso qualsiasi successivamente filtrato (nel caso binaurale si \`e impiegato il set di
\emph{Head Related Transfer Function} del progetto \emph{Listen} dell'\emph{IRCAM} di Parigi);
la coda \`e, invece, prodotta con la tecnica \emph{sinus-burst}, curando di mantenere statisticamente
indipendente ogni canale.\\
Nonostante il calcolo dei parametri acustici mostri ancora un eccesso di energia nella coda
riverberante della risposta all'impulso sintetizzata - peraltro risolvibile, con tutta probabilit\`a,
tramite applicazione di opportuni fattori correttivi alle ampiezze -, il risultato di maggior realismo 
all'ascolto si ritiene pienamente centrato.

%\vspace{0.5cm}
%\begin{figure}[htbp]
%  \centering
%  \includegraphics[width=10cm]{../img/step45270a}
%  \caption{Risposta al gradino $45^{\circ}$ - $270^{\circ}$ (base dei tempi: 500 ms/div).}
%  \label{fig:step45270}
%\end{figure}

\end{document}