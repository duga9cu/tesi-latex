\documentclass[10pt,a4paper]{article}

\usepackage{graphicx}
\usepackage{amsmath}
\usepackage[italian]{babel}
\usepackage{color}
\usepackage{colortbl}

\usepackage{amsmath}
\usepackage{amsthm}
\usepackage{amsfonts}

%lollo packages
\usepackage[font=footnotesize, labelfont=bf]{caption}
\usepackage[hidelinks]{hyperref}
% italian accentuated letters
\usepackage[applemac]{inputenc}


% standard color graphics
\usepackage{array}

%Latin Modern font
\usepackage{lmodern}
\usepackage[T1]{fontenc}
\usepackage{textcomp}

% headers customization
%\usepackage{fancyheadings}
%\usepackage{fancybox}
\usepackage{fancyhdr}
\pagestyle{fancy}

%%%%%%%%%%%%%%%%%%%%%%%%%%%%%% NEW COMMANDS %%%%%%%%%%%%%%%%%%%%%%%%%%%%%%%%

\newcommand{\eigen}{\emph{Eigenmike\textsuperscript{\texttrademark}}\;}
\newcommand{\audacity}{\emph{Audacity\textsuperscript{\textregistered}}\;}


%MARGIN SETTINGS
%
\addtolength{\topmargin}{-2.5cm}
\addtolength{\textheight}{4.5cm}
\addtolength{\textwidth}{2cm}
\addtolength{\hoffset}{-1cm}
%
%PDFLatex SETTING
\pdfpagewidth=\paperwidth
\pdfpageheight=\paperheight
%
\title{\textsc{Elaborazione di mappe acustiche tramite array microfonici}}
\author{Lorenzo Rotteglia}
\date{9 Ottobre 2013}


\begin{document}
%
\pagestyle{empty}
%
%TITLE
\begin{center}
  \LARGE UNIVERSIT\`A DEGLI STUDI DI PARMA\\
\vspace{0.2cm}
\Large 
\vspace{0.3cm}
\normalsize CORSO DI LAUREA MAGISTRALE\\ IN INGEGNERIA INFORMATICA\\
\vspace{0.3cm}
\normalsize ANNO ACCADEMICO 2012 / 2013\\
\vspace{0.3cm}
\rule{4cm}{.1pt}\\
\vspace{0.3cm}
\Large \textsc{Elaborazione di mappe acustiche tramite array microfonici}
\end{center}
%
%\vspace{0.3cm}
\begin{center}
\begin{tabular}{ll}
\parbox[c]{10cm}{\hspace*{0.8cm}\emph{Relatore :}} & \\
\parbox[c]{10cm}{\hspace*{0.8cm}Chiar.mo Prof. Angelo Farina} & \\
\parbox[c]{10cm}{} & \\
\parbox[c]{10cm}{\hspace*{0.8cm}\emph{Correlatore :}}    & \\
\parbox[c]{10cm}{\hspace*{0.8cm}Dott. Ing. Simone Campanini} & \parbox[c]{10cm}{\emph{Tesi di laurea di:}}\\
\parbox[c]{10cm}{\hspace*{0.8cm} }   & \parbox[c]{10cm}{Lorenzo Rotteglia}\\
\end{tabular}
%\vspace{0.4cm}
%\rule{4cm}{.1pt}\\
\end{center}
\vspace{0.8cm}

	%%%%%%%%%%%%%%%%%%%%%%%%%%%%%%%		
%%%%%%%%%%%%%%%% RIASSUNTO %%%%%%%%%%%%%%
	%%%%%%%%%%%%%%%%%%%%%%%%%%%%%%%
	
Il presente progetto di tesi si pone l'obiettivo di ottenere una rappresentazione del campo acustico dinamico di un ambiente (interno o esterno) tramite le registrazioni di una sonda, nella fattispecie un \emph{array microfonico}. 
		Il risultato cercato quindi � quello di ottenere una mappa acustica dinamica, cio� un filmato composto dalla sovrapposizione di:
	
	\begin{description}
	  \item[un videoclip di sfondo] ottenuto da una apposita videocamera corredata agli array microfonici utilizzati, che fornisce un riferimento visuale dell'ambiente circostante;
	  \item[la mappa acustica dinamica] che rappresenta i livelli sonori \emph{istantanei} nel campo acustico attraverso bande cromatiche;
	  \item[eventuali \emph{metadata}] come, ad esempio, le posizioni delle singole capsule microfoniche dell'array, o i valori dei livelli sonori in determinate posizioni di interesse.
	\end{description}  
	
	
	Si � lavorato allo sviluppo di un software che fosse in grado di elaborare registrazioni multicanale prefiltrate, provenienti da un array microfonico generico, congiuntamente a una registrazione video dello stesso evento, in modo da generare in output una mappa sonora dinamica che fornisse una indicazione qualitativa dell'evoluzione del campo acustico. Si tratta di un \emph{plug-in} scritto per l'ambiente di editing audio \audacity.
	
	La mappa acustica generata, di cui si riporta un esempio in figura~\ref{fig:acousticmap}, scaturisce da un'elaborazione a monte, la quale consiste nella sintesi di un array di \emph{microfoni virtuali}\footnote{si tratta di un artificio matematico derivato dalla teoria \emph{Ambisonics} che permette di isolare parte del campo acustico come se fosse stato registrato con un microfono dalla risposta polare arbitraria, direzionato a piacere e posto in posizione coincidente con l'array.} tramite la convoluzione dei segnali \emph{raw} delle capsule con opportuni filtri, calcolati mediante un innovativo approccio numerico.\\	
	
	Questo tipo di \emph{tool} pu� avere numerose applicazioni tecnologiche in quanto � in grado di rendere visibile i campi sonori i quali contengono invece una informazione di tipo uditivo. 
	La potenza di questa operazione sinestetica di traduzione di un'informazione uditiva in una visiva risiede principalmente nella maggior apprezzabilit� delle grandi qualit� di definizione spaziale degli array microfonici. Dal punto di vista applicativo, il sistema sviluppato pu� essere di estrema utilit� nella precisa individuazione di sorgenti sonore, nonch� nella misura delle loro emissioni. 
	
	Basti pensare per esempio ad ambienti di tipo industriale, nei quali spesso il campo acustico � complesso e generato da molteplici e diverse sorgenti la cui individuazione precisa pu� risultare difficoltosa.	
	Un altro esempio calzante riguarda gli ambienti molto ristretti come gli abitacoli di automobili o aereoplani il cui confort � una specifica di progetto che attualmente ricopre un notevole interesse.\\
	In particolare, la principale espansione svolta in questa tesi � stata quella di rendere il software in grado di generare una mappa \emph{dinamica}, riflettendo i cambiamenti del campo acustico attorno alla sonda istante per istante e sovrapponendo questo risultato a un video di sfondo, posto in trasparenza e acquisito mediante telecamere installate appositamente sulle sonde utilizzate e raffigurante l'ambiente stesso di misura.\\

	%%%%%%%%%%%%%%%%%%%%%%%%%%%%%%%% FIGURA MAPPA ACUSTICA %%%%%%%%%%%%%%%%%%%%%%%%%%%%
	\begin{figure}
	  \centering
	  \includegraphics[width =14cm]{../img/acousticmap.png}
	  \caption{Esempio di mappa acustica generata dal plug-in \emph{Microphone Array Analyzer}.\\}
	  \label{fig:acousticmap}
	\end{figure}    
	Come scelta di progetto � stata mantenuta la linea guida di implementare il software in forma di plug-in \audacity, in quanto si tratta di una piattaforma open-source che permette ai suoi \emph{moduli} di accedere a moltissime propriet� dell'host nonch� ai dati del workspace, a differenza di quanto accade con altri standard diffusi come per esempio \emph{VST} e \emph{Audiounits}.

	Il plug-in progettato � funzionante ed altamente interattivo: � possibile infatti realizzare analisi diverse, grazie alla possibilit� di configurare numerose opzioni della mappa, quali la scelta della banda frequenziale a cui limitare l'analisi, l'abilitazione dell'\emph{autoscale}, indicazione delle posizioni dei microfoni virtuali sulla mappa, la scala di colore, la percentuale di trasparenza della mappa sovrapposta al video di background, l'unit� di misura dei livelli mostrati, i valori di livello limite entro i quali calcolare la mappa acustica, la lunghezza del frame audio, la percentuale di \emph{overlap} tra i singoli frame. 
	



\end{document}