\chapter{Descrizione plug-in: un manuale d'uso}
\label{sec:plugindesc}
Le operazioni compiute dal software qui descritto si dividono sostanzialmente in tre parti:
\begin{itemize}
  \item una prima parte di \emph{configurazione}
  \item una parte di \emph{precalcolo}
  \item una parte di \emph{customizzazione} dei risultati tramite l'utilizzo di varie opzioni fornite dalla \emph{GUI}
\end{itemize}
In questo capitolo si vogliono descrivere le possibilit� che l'utente si trova a sfruttare nell'utilizzo del \emph{plug-in} sia in fase di configurazione che in fase di presentazione e analisi dei risultati, mentre la parte di calcolo verr� descritta pi� approfonditamente nel capitolo\ref{sec:internalfunct} alla luce dei parametri utilizzati e presentati ora.


	
	\section{Selezione dell'intervallo temporale da analizzare}
	\label{sec:audioselection}
		Prima ancora di lanciare il \emph{plug-in}, e necessario caricare il file audio della registrazione multicanale acquisita con l'\emph{array microfonico} nell'host \emph{Audacity\textregistered}. 
		Di tutta la registrazione, che pu� in alcuni casi durare anche qualche ora, potrebbe essere necessario selezionare solo una parte, in modo da focalizzarsi nell'analisi su un particolare istante, all'occorrenza anche molto breve. 
		Si pensi per esempio all'analisi delle prime riflessioni in una sala da concerto; analisi nella quale si � interessati a studiare il comportamento di un ambiente sotto l'aspetto degli echi e altri effetti che possono essere fastidiosi per alcuni utilizzi dell'ambiente stesso.
		A questo scopo � necessario selezionare la porzione di audio di interesse, come indicato in figura \ref{fig;audioselection}. 
		Questa e la procedura con la quale  \emph{Audacity\textregistered} gestisce il materiale audio da editare attraverso i \emph{plug-ins} (che nel men� del programma vengono denominati \emph{effects}).
		
		In alternativa � possibile selezionare anche l'intera registrazione, consapevoli che essendo l'analisi compiuta nel \emph{precalcolo} computazionalmente molto onerosa, nonch� esigente per quanto riguarda l'utilizzo delle \emph{risorse}, si va incontro alla possiblit� che il \emph{precalcolo} si interrompa prima di aver analizzato tutti i frame, debitamente all'esaurimento delle risorse a disposizione. 
		In tal caso verr� presentato un messaggio di avviso, come da Figura \ref{fig:outofmemory}, e il programma passer� alla visualizzazione dei soli risultati analizzati.
		Si tenga presente, a titolo di riferimento, che su un normale laptop con processore \emph{dual core} da $2.0 \; GHz$ con $2\;Gb$ di $RAM$, l'elaborazione pi� lunga complessivamente effettuabile consta di circa $60$ frames e impiega $2\;min$ per il \emph{precalcolo}.
		
		Sar� sempre possibile costruire un video a partire dai singoli frames esportati in diverse elaborazioni del plug-in, operando su spezzoni audio consecutivi. 
		
		
		
		
	\section{Finestra di configurazione}
	\label{sec:confdlg}
		[...]\\
		
		
	
	\section{Interfaccia principale}
	\label{sec:MAAdlg}
		[...]\\
		
	
	\section{Esportazione dei risultati}
	\label{sec:export}
		[...]\\
		
		
	


%%%%%%%%%%%%%%%%%%%%%%%%%%%%%%%%%%%%%%%%%%%%%%%%%%%%%%%%%%%%%%%%%%%%%%%%%%%%%%%%%%%%%%%%%%%%%%%%%%%%%%%%%%%%%%%%%%%%%%%%%%%%%%%

%\begin{table}[htbp]
%   \centering
%   \begin{tabular}{l r r r r r r r}
%      $f_\textrm{cb}$ &   125 &   250 &   500 &  1000 &  2000 &  4000 &  8000 \\
%      \hline
%      \emph{SdF}      & -7.96 & -7.80 & -7.56 & -7.34 & -6.92 & -5.88 & -3.92 \\
%      \emph{SdP 1}    & -3.66 & -3.53 & -3.31 & -3.15 & -2.70 & -1.61 &  0.74 \\
%      \emph{SdP 2}    & -1.03 & -0.91 & -0.72 & -0.65 & -0.25 &  0.69 &  3.26 \\
%      \emph{SdP 3}    &  0.23 &  0.36 &  0.53 &  0.57 &  0.95 &  1.80 &  4.57 \\
%      \emph{SdP 4}    &  1.63 &  1.72 &  1.86 &  1.80 &  2.19 &  3.05 &  6.61 \\
%      \emph{SdP 5}   
%   \end{tabular}
%   \caption{$C_{80}$ per un ascoltatore situato a met\`a sala: confronto tra stato di fatto e gli stati di progetto elaborati}
%   \label{tab:c80mid}
%\end{table}

%\begin{table}[htbp]
%   \centering
%   \begin{tabular}{l r r r r r r r}
%      $f_\textrm{cb}$ &   125 &   250 &   500 &  1000 &  2000 &  4000 &  8000 \\
%      \hline
%      \emph{SdF}      & -7.58 & -7.45 & -7.22 & -7.04 & -6.65 & -5.68 & -3.73 \\
%      \emph{SdP 1}    & -5.85 & -5.72 & -5.49 & -5.33 & -4.87 & -3.81 & -1.28 \\
%      \emph{SdP 2}    & -2.29 & -2.17 & -1.95 & -1.85 & -1.42 & -0.42 &  2.33 \\
%      \emph{SdP 3}    & -1.29 & -1.12 & -0.88 & -0.81 & -0.33 &  0.75 &  4.04 \\
%      \emph{SdP 4}    &  0.79 &  0.90 &  1.08 &  1.03 &  1.49 &  2.49 &  6.50 \\
%      \emph{SdP 5}   
%   \end{tabular}
%   \caption{$C_{80}$ per un ascoltatore situato in fondo alla sala: confronto tra stato di fatto e gli stati di progetto elaborati}
%   \label{tab:c80far}
%\end{table}
